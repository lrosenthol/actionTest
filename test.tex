% Options for packages loaded elsewhere
\PassOptionsToPackage{unicode}{hyperref}
\PassOptionsToPackage{hyphens}{url}
\PassOptionsToPackage{dvipsnames,svgnames*,x11names*}{xcolor}
%
\documentclass[
]{article}
\usepackage{lmodern}
\usepackage{amssymb,amsmath}
\usepackage{ifxetex,ifluatex}
\ifnum 0\ifxetex 1\fi\ifluatex 1\fi=0 % if pdftex
  \usepackage[T1]{fontenc}
  \usepackage[utf8]{inputenc}
  \usepackage{textcomp} % provide euro and other symbols
\else % if luatex or xetex
  \usepackage{unicode-math}
  \defaultfontfeatures{Scale=MatchLowercase}
  \defaultfontfeatures[\rmfamily]{Ligatures=TeX,Scale=1}
\fi
% Use upquote if available, for straight quotes in verbatim environments
\IfFileExists{upquote.sty}{\usepackage{upquote}}{}
\IfFileExists{microtype.sty}{% use microtype if available
  \usepackage[]{microtype}
  \UseMicrotypeSet[protrusion]{basicmath} % disable protrusion for tt fonts
}{}
\makeatletter
\@ifundefined{KOMAClassName}{% if non-KOMA class
  \IfFileExists{parskip.sty}{%
    \usepackage{parskip}
  }{% else
    \setlength{\parindent}{0pt}
    \setlength{\parskip}{6pt plus 2pt minus 1pt}}
}{% if KOMA class
  \KOMAoptions{parskip=half}}
\makeatother
\usepackage{xcolor}
\IfFileExists{xurl.sty}{\usepackage{xurl}}{} % add URL line breaks if available
\IfFileExists{bookmark.sty}{\usepackage{bookmark}}{\usepackage{hyperref}}
\hypersetup{
  pdftitle={Markdown Conversion Test},
  pdfauthor={Leonard Rosenthol},
  colorlinks=true,
  linkcolor=Maroon,
  filecolor=Maroon,
  citecolor=Blue,
  urlcolor=blue,
  pdfcreator={LaTeX via pandoc}}
\urlstyle{same} % disable monospaced font for URLs
\usepackage[margin=2.5cm]{geometry}
\usepackage{color}
\usepackage{fancyvrb}
\newcommand{\VerbBar}{|}
\newcommand{\VERB}{\Verb[commandchars=\\\{\}]}
\DefineVerbatimEnvironment{Highlighting}{Verbatim}{commandchars=\\\{\}}
% Add ',fontsize=\small' for more characters per line
\newenvironment{Shaded}{}{}
\newcommand{\AlertTok}[1]{\textcolor[rgb]{1.00,0.00,0.00}{\textbf{#1}}}
\newcommand{\AnnotationTok}[1]{\textcolor[rgb]{0.38,0.63,0.69}{\textbf{\textit{#1}}}}
\newcommand{\AttributeTok}[1]{\textcolor[rgb]{0.49,0.56,0.16}{#1}}
\newcommand{\BaseNTok}[1]{\textcolor[rgb]{0.25,0.63,0.44}{#1}}
\newcommand{\BuiltInTok}[1]{#1}
\newcommand{\CharTok}[1]{\textcolor[rgb]{0.25,0.44,0.63}{#1}}
\newcommand{\CommentTok}[1]{\textcolor[rgb]{0.38,0.63,0.69}{\textit{#1}}}
\newcommand{\CommentVarTok}[1]{\textcolor[rgb]{0.38,0.63,0.69}{\textbf{\textit{#1}}}}
\newcommand{\ConstantTok}[1]{\textcolor[rgb]{0.53,0.00,0.00}{#1}}
\newcommand{\ControlFlowTok}[1]{\textcolor[rgb]{0.00,0.44,0.13}{\textbf{#1}}}
\newcommand{\DataTypeTok}[1]{\textcolor[rgb]{0.56,0.13,0.00}{#1}}
\newcommand{\DecValTok}[1]{\textcolor[rgb]{0.25,0.63,0.44}{#1}}
\newcommand{\DocumentationTok}[1]{\textcolor[rgb]{0.73,0.13,0.13}{\textit{#1}}}
\newcommand{\ErrorTok}[1]{\textcolor[rgb]{1.00,0.00,0.00}{\textbf{#1}}}
\newcommand{\ExtensionTok}[1]{#1}
\newcommand{\FloatTok}[1]{\textcolor[rgb]{0.25,0.63,0.44}{#1}}
\newcommand{\FunctionTok}[1]{\textcolor[rgb]{0.02,0.16,0.49}{#1}}
\newcommand{\ImportTok}[1]{#1}
\newcommand{\InformationTok}[1]{\textcolor[rgb]{0.38,0.63,0.69}{\textbf{\textit{#1}}}}
\newcommand{\KeywordTok}[1]{\textcolor[rgb]{0.00,0.44,0.13}{\textbf{#1}}}
\newcommand{\NormalTok}[1]{#1}
\newcommand{\OperatorTok}[1]{\textcolor[rgb]{0.40,0.40,0.40}{#1}}
\newcommand{\OtherTok}[1]{\textcolor[rgb]{0.00,0.44,0.13}{#1}}
\newcommand{\PreprocessorTok}[1]{\textcolor[rgb]{0.74,0.48,0.00}{#1}}
\newcommand{\RegionMarkerTok}[1]{#1}
\newcommand{\SpecialCharTok}[1]{\textcolor[rgb]{0.25,0.44,0.63}{#1}}
\newcommand{\SpecialStringTok}[1]{\textcolor[rgb]{0.73,0.40,0.53}{#1}}
\newcommand{\StringTok}[1]{\textcolor[rgb]{0.25,0.44,0.63}{#1}}
\newcommand{\VariableTok}[1]{\textcolor[rgb]{0.10,0.09,0.49}{#1}}
\newcommand{\VerbatimStringTok}[1]{\textcolor[rgb]{0.25,0.44,0.63}{#1}}
\newcommand{\WarningTok}[1]{\textcolor[rgb]{0.38,0.63,0.69}{\textbf{\textit{#1}}}}
\usepackage{longtable,booktabs}
% Correct order of tables after \paragraph or \subparagraph
\usepackage{etoolbox}
\makeatletter
\patchcmd\longtable{\par}{\if@noskipsec\mbox{}\fi\par}{}{}
\makeatother
% Allow footnotes in longtable head/foot
\IfFileExists{footnotehyper.sty}{\usepackage{footnotehyper}}{\usepackage{footnote}}
\makesavenoteenv{longtable}
\setlength{\emergencystretch}{3em} % prevent overfull lines
\providecommand{\tightlist}{%
  \setlength{\itemsep}{0pt}\setlength{\parskip}{0pt}}
\setcounter{secnumdepth}{5}
\usepackage{fancyhdr}
\pagestyle{fancy}
\lfoot{Draft Prepared: 15 August 2018}
\rfoot{Page \thepage}

\title{Markdown Conversion Test}
\author{Leonard Rosenthol}
\date{April 16, 2020}

\begin{document}
\maketitle

\newpage{} \toc

\hypertarget{pandoc-test-doc}{%
\section{Pandoc Test Doc}\label{pandoc-test-doc}}

\hypertarget{introduction}{%
\subsection{Introduction}\label{introduction}}

As described in the \href{Glossary.md}{Glossary}, a claim is ``A
JSON-formatted data structure representing the assertions of fact by an
\emph{actor} concerning an \emph{asset} at a specific time and for a
specific reason''. \emph{Claims} in the CAI ecosystem are equivalent to
(and compatible with) a \href{https://www.w3.org/TR/vc-data-model/}{W3C
Verifiable Credential}, however since our claims aren't about people we
don't use the term credential.

\hypertarget{technologies}{%
\subsection{Technologies}\label{technologies}}

\begin{itemize}
\tightlist
\item
  \href{https://tools.ietf.org/html/rfc8259}{JSON}
\item
  \href{https://www.w3.org/TR/json-ld11/}{JSON-LD}
\item
  \href{https://tools.ietf.org/html/rfc7515}{JSON Web Signatures} (JWS)
\item
  \href{https://tools.ietf.org/html/rfc7797}{JSON Web Signature
  Unencoded Payload Option} (JWS-UPO)
\item
  \href{https://tools.ietf.org/html/rfc7519}{JSON Web Token} (JWT)
\item
  \href{https://www.w3.org/TR/vc-data-model/}{Verifiable Credentials}
  (VC)
\item
  \href{https://www.adobe.com/products/xmp.html}{eXtensible Metadata
  Platform} (XMP)
\end{itemize}

\hypertarget{xmp}{%
\subsection{XMP}\label{xmp}}

Every asset, for which a claim is being made, shall contain embedded
XMP. If the asset does not contain XMP at the time a claim is made, the
\emph{claims recorder} shall create it prior to signing the claim. The
\href{https://github.com/adobe/XMP-Toolkit-SDK/}{Adobe XMP Toolkit SDK}
can be used to create and modify XMP in
\href{https://github.com/adobe/XMP-Toolkit-SDK/tree/master/XMPFiles/source/FileHandlers}{various
asset types}.

As defined in the \href{https://www.iso.org/standard/75163.html}{ISO
16684-1 standard}, the XML+RDF serialization of the metadata shall be
uncompressed and can be located starting with the bytes
\texttt{\textless{}?xpacket\ begin=} and ending with the bytes
\texttt{\textless{}?xpacket\ end="w"?\textgreater{}}.

\hypertarget{claim-internals}{%
\subsection{Claim Internals}\label{claim-internals}}

\hypertarget{jwt-claim-set}{%
\subsubsection{JWT Claim Set}\label{jwt-claim-set}}

A claim is defined as a standard JWT claim set
(https://tools.ietf.org/html/rfc7519\#section-4) that also follows the
requirements for a \emph{VC} (6.3.1 of the VC spec) with CAI as the
\emph{\texttt{@context}} and \emph{credentialSubject}. Claims can either
be signed or unsigned. An unsigned claim may contain any values
(\emph{for now}), though it is RECOMMENDED to include the \emph{actions}
that preceded this claim.

\textbf{Example Claim}

\begin{Shaded}
\begin{Highlighting}[]
\FunctionTok{\{}
    \DataTypeTok{"jti"}\FunctionTok{:} \StringTok{"3e061079a991071a5d2dcfd2ee1c6794"}\FunctionTok{,}
    \DataTypeTok{"iss"}\FunctionTok{:} \StringTok{"http://cai.adobe.com"}\FunctionTok{,}
    \DataTypeTok{"iat"}\FunctionTok{:} \DecValTok{1516239022}\FunctionTok{,}
    \DataTypeTok{"vc"} \FunctionTok{:} \FunctionTok{\{}
        \DataTypeTok{"@context"}\FunctionTok{:} \OtherTok{[}
            \StringTok{"https://www.w3.org/2018/credentials/v1"}\OtherTok{,}
            \StringTok{"https://ns.adobe.com/cai"}
        \OtherTok{]}\FunctionTok{,}
            \DataTypeTok{"type"}\FunctionTok{:} \OtherTok{[}\StringTok{"VerifiableCredential"}\OtherTok{,} \StringTok{"AuthenticContent"}\OtherTok{]}\FunctionTok{,}
            \DataTypeTok{"credentialSubject"}\FunctionTok{:} \FunctionTok{\{}
            \DataTypeTok{"actions"}\FunctionTok{:} \OtherTok{[} \FunctionTok{\{} \DataTypeTok{"stEvt:action"}\FunctionTok{:} \StringTok{"filter\_applied"}\FunctionTok{,} \DataTypeTok{"stEvt:when"}\FunctionTok{:} \StringTok{"2020{-}02{-}11T09:00:00"} \FunctionTok{\}} \OtherTok{]}\FunctionTok{,}
            \DataTypeTok{"signature"} \FunctionTok{:} \StringTok{"eyJhbGciOiJIUzI1NiIsInR5cCI6IkpXVCJ9..t3VhQ7QsILDuV\_HNFSMI{-}Fb2FoT7fuzalpS5AH8A9c0"}\FunctionTok{,}
            \DataTypeTok{"url"}\FunctionTok{:} \StringTok{"https://cai\_resolver.adobe.com/A1B2C3D4E5"}\FunctionTok{,}
            \DataTypeTok{"parent\_url"}\FunctionTok{:} \StringTok{"https://cai\_resolver.adobe.com/123456789123456789000"}
        \FunctionTok{\}}
    \FunctionTok{\}}
\FunctionTok{\}}
\end{Highlighting}
\end{Shaded}

The JWT claim set shall include: - \emph{jti} - a unique identifier for
the JWT. (NOTE: as per VC, this is equivalent to the VC \texttt{id}
property) - \emph{iss} - identifies the actor that issued the JWT as a
case-sensitive string containing a StringOrURI. (NOTE: as per VC, this
is equivalent to the VC \texttt{issuer} property) - \emph{iat} -
identifies the time at which the JWT was issued. Its value MUST be a
number expressing a NumericDate - \emph{vc} - this is the verifiable
credential itself, which is a valid JSON-LD object including \emph{type}
and \emph{credentialSubject}.

\hypertarget{adobes-view-of-did}{%
\subsection{Adobe's view of DID}\label{adobes-view-of-did}}

\hypertarget{overview}{%
\subsubsection{Overview}\label{overview}}

The W3C specification for DID defines them as:

\begin{quote}
Decentralized identifiers (DIDs) are a new type of identifier to provide
verifiable, decentralized digital identity. These new identifiers are
designed to enable the controller of a DID to prove control over it and
to be implemented independently of any centralized registry, identity
provider, or certificate authority.
\end{quote}

\hypertarget{command-line-parameters}{%
\subsection{Command Line Parameters}\label{command-line-parameters}}

\hypertarget{standard-parameters}{%
\subsubsection{Standard Parameters}\label{standard-parameters}}

\begin{longtable}[]{@{}ll@{}}
\toprule
\begin{minipage}[b]{0.47\columnwidth}\raggedright
Param\strut
\end{minipage} & \begin{minipage}[b]{0.47\columnwidth}\raggedright
Description\strut
\end{minipage}\tabularnewline
\midrule
\endhead
\begin{minipage}[t]{0.47\columnwidth}\raggedright
\texttt{-h}\strut
\end{minipage} & \begin{minipage}[t]{0.47\columnwidth}\raggedright
print help information\strut
\end{minipage}\tabularnewline
\begin{minipage}[t]{0.47\columnwidth}\raggedright
\texttt{-\/-help}\strut
\end{minipage} & \begin{minipage}[t]{0.47\columnwidth}\raggedright
print help information\strut
\end{minipage}\tabularnewline
\begin{minipage}[t]{0.47\columnwidth}\raggedright
\texttt{-o\ {[}FILE\textbar{}DIR{]}}\strut
\end{minipage} & \begin{minipage}[t]{0.47\columnwidth}\raggedright
The explicit filename to save to OR a directory where logically named
output will be placed\strut
\end{minipage}\tabularnewline
\begin{minipage}[t]{0.47\columnwidth}\raggedright
\texttt{-\/-log\ {[}FILE{]}}\strut
\end{minipage} & \begin{minipage}[t]{0.47\columnwidth}\raggedright
Instead of logging to stdout, write to a specified file instead\strut
\end{minipage}\tabularnewline
\begin{minipage}[t]{0.47\columnwidth}\raggedright
\texttt{-\/-pages}\strut
\end{minipage} & \begin{minipage}[t]{0.47\columnwidth}\raggedright
range/list of pages to be processed (eg. 1-5, 2,4,7). NOTE: page range
is normally one-based unless --zero is also used, then it's
zero-based\strut
\end{minipage}\tabularnewline
\bottomrule
\end{longtable}

\end{document}
